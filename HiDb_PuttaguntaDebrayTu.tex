\documentclass[10pt]{article}

% Preamble to the document, to avoid cluttering it up

\title{HiDb: A Haskell In-Memory Relational Database}
\author{
	\textsc{Rohan Puttagunta}
		\qquad
	\textsc{Arun Debray}
		\qquad
	\textsc{Susan Tu}
		\mbox{}\\ %
		\\
		CS240H\\
		\mbox{}\\ %
		\normalsize
			\texttt{rohanp}
		\textbar{}
			\texttt{adebray}
		\textbar{}
			\texttt{sctu}
		\normalsize
			\texttt{@stanford.edu}
}
\date{\today}

%\documentclass{acmconf}
%\usepackage[pdftex]{graphicx}

\usepackage{listings}
\usepackage{xcolor}
%\usepackage[paper=a4paper,dvips,top=2cm,left=1.5cm,right=1.5cm,
%	foot=3cm,bottom=3cm]{geometry}


% listings magic!
\definecolor{DefaultColor}{HTML}{00428C}
\definecolor{CommentColor}{HTML}{60A0B0}
\definecolor{BackgroundColor}{HTML}{E4E5E7}
\definecolor{FrameColor}{HTML}{95ABD0}
\definecolor{NumberColor}{HTML}{40A070}
\definecolor{IDColor}{HTML}{06278E}
\definecolor{KeywordColor}{HTML}{007020}
\definecolor{TypeColor}{HTML}{902000}


% Courtesy of% http://tex.stackexchange.com/questions/34896/coloring-digits-with-the-listings-package
\newcommand\digitstyle{\color{NumberColor}}\makeatletter\newcommand{\ProcessDigit}[1]{%
  \ifnum\lst@mode=\lst@Pmode\relax%
   {\digitstyle #1}%
  \else
	#1%
  \fi
}
\makeatother
\lstset{literate=
	{0}{{{\ProcessDigit{0}}}}1
	{1}{{{\ProcessDigit{1}}}}1
	{2}{{{\ProcessDigit{2}}}}1
	{3}{{{\ProcessDigit{3}}}}1
	{4}{{{\ProcessDigit{4}}}}1
	{5}{{{\ProcessDigit{5}}}}1
	{6}{{{\ProcessDigit{6}}}}1
	{7}{{{\ProcessDigit{7}}}}1
	{8}{{{\ProcessDigit{8}}}}1
	{9}{{{\ProcessDigit{9}}}}1
}

\lstset{
	backgroundcolor=\color{BackgroundColor},
	basicstyle=\color{DefaultColor}\footnotesize\tt,
	breakatwhitespace=false,
	breaklines=true,
	commentstyle=\color{CommentColor}\textsl,
	frame=true,
	frameshape={yyy}{y}{y}{yyy},
	identifierstyle=\color{IDColor},
	keywordstyle=\color{KeywordColor},
	language=Haskell,
	emph={Bool, Char, String, IO, Maybe, Either, Ordering, Int, Integer, Ratio, Float, Double, Complex, TVar, MVar, STM},
	emphstyle=\color{TypeColor},
	rulecolor=\color{FrameColor},
	rulesepcolor=\color{FrameColor},
	showstringspaces=false,
	stringstyle=\color{FrameColor}, % apparently
	tabsize=2
}

% end of listings magic

\usepackage[margin=2cm]{geometry}
\usepackage{float}
\usepackage{multicol}
\usepackage[pdftex]{graphicx}
\usepackage[english]{babel}
\usepackage{sidecap}
\usepackage[font=small,labelfont=bf]{caption}
\usepackage[raggedright]{titlesec}
\usepackage{hyperref}

\makeatletter
\newenvironment{tablehere}
  {\def\@captype{table}}
  {}

\newenvironment{figurehere}
  {\def\@captype{figure}}
  {}
\makeatother


\begin{document}

\maketitle

\begin{abstract}
We describe our experience implementing an in-memory relational database in Haskell that supports the standard CRUD (create, read, update, delete) operations while providing the requisite ACID (atomicity, consistency, isolation, durability) guarantees. We rely on Haskell's STM module to provide atomicity and isolation. We use a combination of STM, Haskell's type system, dynamic type-checking in order to enforce consistency. We implement undo-redo logging and eventual disk writes to provide durability. We also provide a Haskell library which clients can use to connect to and send transactions to the database. We found that while the STM module greatly eased the implementation of transactions, the lack of support for de-serializing data into a dynamic type was not ideal. 
 
\end{abstract}

\vspace{5mm}
\begin{multicols}{2}

\section{Introduction} 

\section{Database Operations}
\subsection{Supported Syntax}
\label{simplifying_parsing_assumptions}
The database operations that we support are \texttt{CREATE TABLE}, \texttt{DROP TABLE}, \texttt{ALTER TABLE}, \texttt{SELECT}, \texttt{INSERT}, \texttt{SHOW TABLES}, \texttt{UPDATE}, and \texttt{DELETE}.  We support the following syntax for specifying these operations:
TODO ARUN \\\\
\subsection{Implementation}
Each operation is implemented as a Haskell function.  Operations which make changes to the database, should they succeed, should return lists of \texttt{LogOperation}s, where the \texttt{LogOperation} datatype represents log entries and we use different constructors for different types of entries (see table x). Since we cannot perform IO from within STM, the calling function is responsible for actually writing these \texttt{LogOperation}s, which are instances of \texttt{Read} and \texttt{Show} to allow for easy serializability and de-serializability, to the in-memory log. If the operation was malformed (for example, the referenced table does not exist, or the user failed to specify the value for a column for which there is no default value), then we return some error message to the user. For operations such as \texttt{SELECT} that do not modify the database, we return a string that can be written back to the user. \\\\
In our Haskell implementation of \texttt{SELECT}, we choose to return a \texttt{Table} rather than a \texttt{String} because while we did not implement this functionality in this verison of HiDb, in theory we perform futher operations involving the returned table. We also chose to make use of a \texttt{Row} datatype, which is a wrapper around a \texttt{Fieldname -> Maybe Element} function. This allows us to use functions of the type \texttt{Row -> STM(Bool)} as the condition for whether a row of a table should be deleted or updated. It also allows us to express an update as \texttt{Row -> Row}. 

\section{Data Structures}

\subsection{Concurrency} 
\noindent\begin{minipage}{.45\textwidth}
\begin{lstlisting}[caption=The data structures used to build up a table.,frame=tlrb, breaklines=true]
data Table 
  = Table { rowCounter :: Int 
          , primaryKey :: Maybe Fieldname 
          , table :: Map Fieldname Column
		  }

data Column 
  = Column { default_val :: Maybe Element
           , col_type :: TypeRep
           , column :: TVar(Map RowHash (TVar Element))
           } -- first element is default value

data Element = forall a. (Show a, Ord a, Eq a, Read a, Typeable a) => 
  Element (Maybe a) -- Nothing here means that it's null
\end{lstlisting}
\end{minipage}\hfill

Note that we needed to use the \texttt{Existential Quantification} language extension in our definition of \texttt{Element}, which allows us to put any type that is an instance of \texttt{Show}, \texttt{Ord}, \texttt{Eq}, \texttt{Read}, and \texttt{Typeable}.  Note that this means that we do not statically enforce the fact that every column should contain elements of just one type. It seems that it is not possible to statically enforce this in Haskell, since when the user attempts an insert, there is no way to know statically (since the user only provides the input at runtime) whether this input will be of the same type as the other elements in the column. We therefore must enforce the type of columns at run-time (specifically, in the parsing stage, which will be discussed later in the paper) by keeping track of one \texttt{TypeRep} per column. 

We use multiple layers of \texttt{Tvar}s in order to allow operations that touch different parts of the database to proceed at the same time. For example, having each \texttt{Table} be stored in a \texttt{Tvar} means that if client $A$ updates a row in one table, client $B$ can concurrently update a row in another table. Moreover, consider what happens if we attempt changes that affect the database at different levels of granularity: Suppose we concurrently attempt two operations, $O_1$ and $O_2$, where $O_1$ is the insertion of a new column into a table and $O_2$ is the updating of a value in an existing column in the same table. If $O_1$ completes first, then $O_2$ will have to be retried because the pointer (which had to be read for $O_2$ to occur) in the \texttt{Tvar} surrounding the \texttt{Table} type will have changed. However, if $O_2$ completes first, then $O_1$ will complete without issue because the only $Tvar$s that $O_2$ changed are ones that did not need to be read for $O_1$. 


\section{Parsing}
% Section on parsing.

Though parsing wasn't the prime directive of this project, it was still a major aspect of it: in order to provide a usable end program, we needed a server that can respond to queries and manage a database.

Upon connecting to a client, this server devotes a thread to listening to instructions from a client, which are received in blocks. Since there may be multiple clients acting at the same time, as well as possible other threats to durability (e.g. power outages), it would be optimal for all of these commands to be run in one atomic block, so that they can be easily undone and redone.

However, a few commands alter the structure of the database itself, i.e. \verb+CREATE TABLE+, \verb+DROP TABLE+, and \verb+ALTER TABLE+. These commands change the types of fields or the tables themselves, and thus should go in their own atomic actions.

Thus, the first step in parsing is to split the client's block of actions into sections that can be executed atomically: as many commands as possible that don't alter the structure of the database, followed by one that does, then as many as possible that don't, and so on. The following code provides a simplified implementation of this.
\lstinputlisting[caption=Grouping the client's issued commands.]{listing2.hs}
Once this is done, the server has blocks of commands that can be issued atomically. This allows us to take advantage of the STM framework provided by Haskell, using the \verb+atomically+ function to convert all of the atomic actions on the database into one \verb+IO+ action that will execute atomically.

The return type of these operations is of the form \verb+STM (Either ErrString [LogOperation])+, and these two cases are handled differently:
\begin{itemize}
	\item If the database operation returns \verb+Left err+, then the rest of the block should stop executing, and the error is reported to the user.
	\item If the database operation succeeded, returning \verb+Right logOp+, then the resulting statements should be written to the log, to power the undo/redo logging of the whole database.
\end{itemize}
Here's an example of how this was handled in the code.
\lstinputlisting{listing3.hs}
That atomicity was the easiest aspect of the parser/server was one of the primary advantages of using Haskell.

Finally, the \verb+parseCommand+ function pattern-matches on the different possible commands and chooses the correct operation to execute. This is where the bulk of the explicit parsing was done; however, in order to simplify this function, we made some simplifying constraints on the subset of SQL accepted by this parser, as detailed in Section~\ref{simplifying_parsing_assumptions} above.

One trick in parsing was that the database is strongly typed, but the client sends over strings; in theory, the server has to read data from strings into an arbitrary type specified at runtime. Haskell's type system does not make this very easy, and we found it much easier to restrict to several basic types and pattern-match against a \verb+TypeRep+ stored along with every database column. Then, there are only a few types to check, and these can be done without too much code.

The types we used are \verb+Int+, \verb+Real+ (backed by a \verb+Double+), \verb+Bool+, and \verb+Char+ and bit arrays; these were chosen to represent the fundamental SQL types. It would not be hard to add more types, since this just involves adding one step to each pattern-matching function for the types.

\section{Durability}

\section{Summary}


\end{multicols}

\begin{thebibliography}{11}

\bibitem{harris} \url{http://research.microsoft.com/pubs/67418/2005-ppopp-composable.pdf}

\end{thebibliography}
\end{document}
